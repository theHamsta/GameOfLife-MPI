Existem vários maneiras de representar o estado atual do tabuleiro. Elas têm varias vantagens e desvantagens considerando consumação de memoria, utilização dos \textit{caches}, número de acessos necessários para calcular a próxima geração e também no rendimento do algoritmo a ser utilizado. Nesta seção, śo se apresentarão umas dessas possibilidades. Uma discussão mais detalhada das suas propriedades para diferentes aspectos do algoritmo seguirá nas seções seguintes.\\

\subsubsection{Representação Explícita}
Na representação mais simples, grava-se os estados das células numa matriz de tipo \textit{byte}. Cada \textit{byte} pode assumir os valores 0 e 1, ou seja vivo e morto. Obviamente, este método não muito é eficiente, já que sete \textit{bits} permanecem não usados. Além disso, precisa-se de nove acessos à memória\footnote{Ou Três, reutilizando \textit{bytes} já lidos} para determinar e um acesso para gravar o próximo estado de cada célula. Para evitar ditas desvantagens, pode-se usar um \textit{bit} só por estado a ser gravado. De este modo, é possível ler mais que uma célula à vez, dependendo do tamanho dos registros do CPU. Outra ventagem é a possibilidade de usar instruções lógicas  para contar o numero de vizinhos vivos tal como usar os \textit{bits} diretamente como índex de uma \textit{Look Up Table (LUT)} para averiguar o destino de vários células à vez.\\

\subsubsection{Representação Explícita com Informações sobre a Vizinhança}
Embora um \textit{bit} por célula seja o modo mais compacto de gravar o tabuleiro, existem técnicas que introduzem mais redundância para acelerar possíveis algoritmos. Uma possibilidade é armazenar vizinhança junta com os estados atuais. Isso basa-se na heurística que a maioria das células não muda numa ronda do jogo. De tal forma, pode-se avançar rapidamente por grandes partes do tabuleiro utilizando poucas operações simples. Por outro lado, será preciso uma atualização das vizinhanças em caso das células que mudam o seu estado. Também cresce o consumo de memória: 9 bit por cada célula, devido ao armazenamento redundante.\\


\subsubsection{Representações Implícitas}

Por outro lado, existem representações que se limitam a armazenar só as células vivas. Isso tem tanto vantagens no consumo de memória como na eficiência de alguns algoritmos. Possibilidades para isso são entre outros formatos criados para matrizes esparsas como o formato CRS (\textit{compressed row storage }) ou o uso de \textit{quad trees} que mantêm uma coleção de colonias de bactérias. Só se vai discutir aqui o uso de listas, já  que o desenho desta implementação foi criada originalmente para ser logo  melhorada por elas. \\

Listas para cada linha de células podem consistir de pares de entradas: a coluna duma bactéria viva e o endereço da próxima nesta fila. Também seria possível gravar triples com a terceira entrada de uma pequena área local como a que será apresentada em seção \ref{sec:representacao_usada}. O uso de listas é vantajoso porque podem ser tratados por fila com a possibilidade da introdução fácil de entradas na fila anterior. 


\subsubsection{Representação Usada}
\label{sec:representacao_usada}

\begin{figure}[h]
\centering

\begin{tikzpicture}[darkstyle/.style={draw,fill=gray!40,minimum size=20},lightstyle/.style={draw,fill=gray!20,minimum size=20}, whitestyle/.style={draw,minimum size=20}]
  \foreach \x in {0,...,5}
    \foreach \y in {0,...,4} 
{\pgfmathtruncatemacro{\label}{\x - 6 *  \y +24}
\pgfmathtruncatemacro{\xnew}{5-\x}
\pgfmathtruncatemacro{\ynew}{4-\y}

\ifthenelse{\x > 0 \and \x < 5 \and \y > 0 \and \y < 4}{
       \node [darkstyle]  (\x\y) at (0.8*\xnew,0.8*\ynew)
       }{
       \node [lightstyle]  (\x\y) at (0.8*\xnew,0.8*\ynew)
       } {\label};} 
       
   \node[whitestyle] (30) at (-2,0) {30};
   
   \node[whitestyle] (31) at (-2.8,0) {31};
   \node[text width=1cm, align=center] (unused) at (-2, 1.2) {\textsf{Não usado}};
   \node (waschanged) at (-2.8, 2) {\textsf{Indicador de Mudança}};
   \node[fill=gray!20] (viz) at (6, 2) {\textsf{Vizinhança}};
   \node[fill=gray!40] (cel) at (5.5, 0.3) {\textsf{Células}};
   
   \draw[->, line width=2] (unused) -- (30);
   \draw[->, line width=2] (waschanged) -- (31);
    \draw[->, line width=2] (viz) -- (01.east);
   \draw[->, line width=2] (cel) -- (22.south east);
   
\end{tikzpicture}
\caption{Representação de um grupo de doze células com 32 bit }
\label{fig:representacao}
\end{figure}

Para a implementação aqui apresentada, gravaram-se grupos de 3x4 células (3 linhas horizontais com 4 colunas) juntas com a sua vizinhança em uma palavra de 32 bit, segundo figura \ref{fig:representacao}. O tabuleiro está representado explicitadamente, aliás como matriz de células vivas e mortas. As razões por essa escolha serão explicados na próxima seção.


\label{sec:distribuicao}

Usa-se o MPI em situações em que tamanho do problema ultrapassar a capacidade de uma maquina a tratar o mesmo. Portanto, é importante achar um jeito para dividir o problema em subproblemas que podem ser solucionados separadamente de forma paralela com o um mínimo de comunicação e sincronização entre os diferentes processos. Além disso, deve-se realizar a comunicação inevitável de forma paralela à computações. \\

No caso de uma representação explícita das células, aquela formação de subunidades pode ser feito facilmente por divisão do tabuleiro global em fatias o retângulos locais do mesmo tamanho. A desvantagem das fatias, em comparação com quadrados, é uma superfície muito maior, que precisa de intercâmbio de dados entre os processos. Porém, a comunicação é mais simples, já que só dois vizinhos precisam ser considerados em vez de oito.\\

A implementação realiza uma divisão estática em retângulos. Os bordos criados por o algoritmo local na seção \ref{sec:algoritmo} são utilizados como mensagem a ser enviada aos respetivos vizinhos. Isso é o método que requer menos computação. Versões futuros do programa poderiam compactar aquelas mensagens de tal forma que só se transmita as colunas ou filas dos unidades que tem mudado. Outra melhoria pode ser realizado em enviar células interiores, como as mais cercanas ao borde, com o objetivo de executar mais que uma iteração entre as mensagens.\\

Como já mencionado na seção anterior, as células no bordo podem ser processados antes das interiores. Deste modo, é possível enviar mensagens pouco depois do inicio da uma iteração. Um momento no qual o vizinho ainda não precisa desta informação. De tal forma, ele pode calcular em paralela quando está recebendo as mensagens.\\

Em realizações mais dinâmicas que se basem em manter diferentes colonias de bactérias, precisa-se de processo \textit{master} que impede incongruências em caso de encontro de duas colonias. Os processos poderiam enviar ao \textit{master} uma lista da retângulos mínimos que contêm cada colonia. O \textit{master}, pela sua parte, ordenaria que um dos processos afetados, em caso de uma colisão de colonias, juntasse elas. Outra possibilidade seria que cada processo conhece os seus  vizinhos diretos, por exemplo por formar parte de um \textit{quad tree} distribuído, e só tenta comunicar com eles. As duas soluções requerem de um mecanismo de balançamento de carga que se vai discutir na próxima seção.


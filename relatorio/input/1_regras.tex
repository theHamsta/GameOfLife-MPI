O Jogo da Vida é um autômato celular criado em 1970 por John Horton Conway \cite{wikipedia}. Simula uma colonia de bactérias num tabuleiro de células retangulares.\\

As regras desse jogo foram construídas simples. Cada bactéria está representada por uma célula que pode ter dois estados: viva ou morta. Uma vez inicializado o tabuleiro (aleatoriamente ou com configurações determinadas previamente), calcula-se em cada ronda do jogo uma nova geração de bactérias usando os estados das células na geração anterior.\\

Se uma bactéria morre, nasce ou fica no seu estado atual, depende da sua vizinhança, ou seja as oito células no entrono  (abaixo, em cima, à esquerda,  à direita e nos sentidos diagonais) da célula a ser calculada. Segundo \cite{wikipedia} as regras são as seguintes:
\begin{enumerate}

 \item Qualquer célula viva com menos de dois vizinhos vivos morre de solidão.
 \item   Qualquer célula viva com mais de três vizinhos vivos morre de superpopulação.
 \item   Qualquer célula morta com exatamente três vizinhos vivos se torna uma célula viva.
 \item   Qualquer célula viva com dois ou três vizinhos vivos continua no mesmo estado para a próxima geração.

\end{enumerate}

